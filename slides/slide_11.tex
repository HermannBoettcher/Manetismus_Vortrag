\documentclass[../defence.tex]{subfiles}
\begin{document}

  \begin{frame}{Annormaler ganzzahliger Quanten Hall Effekt (QHE) in Graphen II}
        \begin{alertblock}{Falscher Ansatz}
          \begin{itemize}
            \item Alle Landauniveaus beteiligt\\
            $\rightarrow$ $\delta E_\mathrm{inc}=\pm 4neV_\mathrm{H}$\\
            $\rightarrow$ $I_\mathrm{inc}\pm 4(e^2/h)V_\mathrm{H}$ $\rightarrow$ $\sigma_ {xy,\mathrm{inc}}=\frac{I}{V_\mathrm{H}}=\pm 4ne^2 / h$
            \item \faBolt$\quad$ Hall plateau bei $n=0$, wenn $\mu$ auf dem Dirac-Punkt liegt
          \end{itemize}
        \end{alertblock}
        \begin{exampleblock}{Richtiger Ansatz}
          \begin{itemize}
            \item
          \end{itemize}
        \end{exampleblock}
    \note{
      \begin{itemize}
        \item Alle Landayniveaus beteiligt $\rightarrow$ 4 wegen der zweifachen Spinentartung und der zweifachen Quasispinentartung
        \item Nach vorheriger Ausführung existiert ein Landauniveau für $n=0$, also kann da kein Hallplateau liegen
      \end{itemize}
    }
  \end{frame}

\end{document}
