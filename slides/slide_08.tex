\documentclass[../defence.tex]{subfiles}
\begin{document}

  \begin{frame}{Quanten Hall Effekt (QHE) I}
    \begin{block}{Ganzzahliger Quanten Hall Effekt}
      \begin{itemize}
        \item charakteristisch für Monolagen-Graphen
        \item Äquidistante Stufen  der Hall-Leitfähigkeit $\sigma_{xy}$
        \item Verschiebung um $1/2$ im Vergleich zum normalen QHE
        \item $\sigma_{xy}=\pm 4e^2/h (N+1/2)$, $N$-Landauniveau
      \end{itemize}
    \end{block}
    \begin{alertblock}{Quantisierung des elektronischen Spektrums}
      \begin{equation*}
        E_N=\pm v_\mathrm{F}\sqrt{2e \hbar BN}
      \end{equation*}
      \textbf{Es existiert ein quantisiertes Niveau bei $N=0$, geteilt von Elektronen und Löchern.}
    \end{alertblock}
    \note{
      \begin{itemize}
        \item $\pm$ respektive für Elektron oder Loch
      \end{itemize}
    }
  \end{frame}

\end{document}
