\documentclass[../defence.tex]{subfiles}
\begin{document}
  \section{Landauniveaus}

  \begin{frame}{Landauniveaus in Graphen I}
    \begin{block}{\textit{Normale} Landauniveaus}
      \begin{equation*}
        E_n=\left( n+\frac{1}{2}\right)\hbar \omega_c
      \end{equation*}
    \end{block}
    \begin{block}{Monolage}
      \begin{equation*}
        E_n=\mathrm{sgn}(n)\sqrt{2e\hbar v_f^2 |n| B}
      \end{equation*}
    \end{block}
    \begin{block}{Doppellage}
      \begin{equation*}
        E_n=\mathrm{sgn}(n)\hbar \omega_c\sqrt{|n|(|n|+1)}
      \end{equation*}
    \end{block}
    \note{
    \begin{itemize}
      \item \textit{Normales} Landauniveau: Translationsenergie in Feldrichtung weggelassen!
      \item Energie relativ zum Dirac-Punkt gemessen (da, wo Leitungsband und Valenzband sich berühren)
      \item Bei Monolage und Doppellage: $E_n(n=0)=0$! Existenz eines null Energie Niveaus
    \end{itemize}
    }
  \end{frame}

\end{document}
