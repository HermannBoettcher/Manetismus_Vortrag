\documentclass[../defence.tex]{subfiles}
\begin{document}

  \begin{frame}{Quanten Hall Effekt in Graphen I}
    \begin{block}{Besonderheiten in Graphen}
        \pause
        \begin{itemize}
          \item Observierbar bei Raumtemperatur
          \pause
          \item Masselose relativistische Teilchen als Ladungsträger
        \end{itemize}
    \end{block}
    \pause
    \begin{block}{Ursprung}
      \pause
      \begin{itemize}
        \item Gebundene $e^-$ im C-Atom $\rightarrow$ nicht relativistisch
        \pause
        \item $e^-$ im periodischen Potential der Kristallstruktur von Graphen $\rightarrow$ relativistisch
        \pause
        \item Quasiteilchen; durch Dirac-Gleichung beschrieben
        \pause
        \item $v_\mathrm{F}$ statt $c$
      \end{itemize}
    \end{block}
    \note{
    \begin{itemize}
      \item 10 mal höhere Temperatur als bisher in anderen Materialien observiert
      \item Dirac-Gleichung, nicht Schrödinger-Gleichung beschreibt die elektrischen Eigenschaften am einfachsten
      \item Quasiteilchencharakter vergleichbar mit geladenen Neutrinos
      \item Anstelle der Lichtgeschwindigkeit $c$ tritt die Fermi-Geschwindigkeit $v_\mathrm{F}$ der $e^-$
    \end{itemize}
    }
  \end{frame}

\end{document}
