\documentclass[../defence.tex]{subfiles}
\begin{document}
  \section{QED}

  \begin{frame}{Quantenelektrodynamik in Graphen I}
    \pause
    \begin{block}{Besonderheiten in Graphen}
        \pause
        \begin{itemize}
          \item Gebundene $e^-$ im C-Atom $\rightarrow$ nicht relativistisch
          \pause
          \item $e^-$ im periodischen Potential der Kristallstruktur von Graphen $\rightarrow$ \textbf{relativistisch}
          \pause
          \item Masselose relativistische Teilchen als Ladungsträger (bei kleinen Energien!)
          \pause
          \item Quasiteilchen; durch Dirac-Gleichung beschrieben
          \pause
          \item $v_\mathrm{F}$ statt $c$
        \end{itemize}
    \end{block}
    \note{
    \textbf{Besonderheiten}
    \begin{itemize}
      \item Gebundene $e^-$: Nicht relativistische
      \item Im Potential des Kristallgitters: Relativistisch
      \item Masseloste relativistische Ladungsträger\\ $\rightarrow$
            Dirac-Gleichung, nicht Schrödinger-Gleichung beschreibt die elektrischen Eigenschaften am einfachsten
      \item Quasiteilchencharakter vergleichbar mit geladenen Neutrinos
      \item Anstelle der Lichtgeschwindigkeit $c$ tritt die Fermi-Geschwindigkeit $v_\mathrm{F}$ der $e^-$\\
            $\rightarrow$ Weil QED Phenomäne proportional zur Geshwindigkeit der Teilchen Effekte in Graphen verstärkt!
    \end{itemize}
    }
  \end{frame}

\end{document}
