\documentclass[../defence.tex]{subfiles}
\begin{document}

  \begin{frame}{Graphen - Einführung}
      \begin{itemize}
        \item 2D Monoschicht aus Kohlenstoffatomen in Bienenwabenstruktur
        \item Grundbaustein aller andersdimensionalen Graphitstrukturen
        \item Zunächst für "akademisches Material gehalten" (thermodynamisch instabil)
        \item 2004 als stabile Strukturen entdeckt
        \item Exeptionell hohe kristalline und elektronische Qualität
      \end{itemize}
      \note{
        \begin{itemize}
          \item Zwei überlappende Dreiecksgitter
          \item Bilder folgen gleich
          \item Schmelztemperatur von Dünnfilmen sinkt rapide mit kleiner werdenden Dicke
          \item Erklärung: Wegen hoher interatomarer Bindungsenergie nicht anfällig für thermische Dislokationen und andere Kristalldefekte; \\
          Lecht gekrumpelt $\rightarrow$ Elastische Energie aber Unterdrückung thermischer Vibrationen
          \item Ladungsträger: Masselose Dirac-Fermionen
        \end{itemize}
            }
  \end{frame}

\end{document}
