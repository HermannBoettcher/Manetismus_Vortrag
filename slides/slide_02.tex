\documentclass[../defence.tex]{subfiles}
\begin{document}

  \begin{frame}{Graphen - Einführung}
    \pause
      \begin{itemize}
        \item 2D Monoschicht aus Kohlenstoffatomen in Bienenwabenstruktur
        \pause
        \item Grundbaustein aller andersdimensionalen Graphitstrukturen
        \pause
        \item Zunächst für "akademisches Material gehalten" (thermodynamisch instabil)
        \pause
        \item 2004 als stabile Strukturen entdeckt
        \pause
        \item Exeptionell hohe kristalline und elektronische Qualität
      \end{itemize}
      \note{
        \begin{itemize}
          \item 2D Monoschicht aus C-Atomen - Zwei überlappende Dreiecksgitter
          \item Grundbaustein von Graphitstrukturen - Bilder folgen gleich
          \item Akademisches Material - Schmelztemperatur von Dünnfilmen sinkt rapide mit kleiner werdenden Dicke
          \item In stabiler Form entdeckt - Erklärung: Wegen hoher interatomarer Bindungsenergie nicht anfällig für thermische Dislokationen und andere Kristalldefekte; \\
          Lecht gekrumpelt $\rightarrow$ Elastische Energie aber Unterdrückung thermischer Vibrationen
          \item Exeptionelle kristalline und elektronische Qualität - Ladungsträger: Masselose Dirac-Fermionen
        \end{itemize}
            }
  \end{frame}

\end{document}
