\documentclass[../defence.tex]{subfiles}
\begin{document}

  \begin{frame}{Chiralität}
    \pause
    \begin{block}{Definition}
      Zerlegung von Dirac-Spinoren in orthogonale Zustände, die unter Paritätsoperationen ineinander übergehen.
    \end{block}
    \pause
    \begin{block}{Von Graphen}
      \pause
      \begin{itemize}
        \item Projektion von $\vec\sigma$ auf $\vec k$
        \pause
        \item Positiv für Elektronen, negativ für Löcher
        \pause
        \item Kontext: $k$ Elektronen, $-k$ Löcher
      \end{itemize}
    \end{block}
  \end{frame}
  \note{
        \textbf{Parität}
        \begin{itemize}
          \item Transformation $(t,.x,y,z)\rightarrow (t,-x,-y,-z)$
          \item Parität einer Größe \textit{positiv}, falls invariant
          \item Parität einer Größe \textit{negativ}, falls Vorzeichenwechsel
        \end{itemize}
  }

\end{document}
