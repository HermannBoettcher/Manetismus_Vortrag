\documentclass[../defence.tex]{subfiles}
\begin{document}

  \begin{frame}{Quantenelektrodynamik in Graphen II}
    \begin{block}{Der Hamiltonian für eine Monolage}
      \pause
      $e^-$ entstammen einem der beiden Sub(-dreiecks-)gitter der Bienenwabenstruktur\\
      \pause
      $\rightarrow$ \textit{Pseudospin} wird eingeführt\\
      \pause
      \begin{equation*}
        \OP{H}_1=\hbar v_\mathrm{F}
          \begin{bmatrix}
            0 & k_x - ik_y \\
            k_x+ ik_y & 0
          \end{bmatrix}=\hbar v_\mathrm{F} \vec{\sigma}\vec{k}
      \end{equation*}
    \end{block}
    \pause
    \begin{alertblock}{Wert von Graphen für die Forschung}
      Quantenelektrodynamische Phenomäne $\frac{c}{v_\mathrm{F}}\approx 300$ mal stärker in Graphen als bisher in anderen Materialien observiert!
    \end{alertblock}
    \note{
          \textbf{Hamiltonian für Monolage}
          \begin{itemize}
            \item Bienenwabenstruktur entspricht zwei überlagerten Dreiecksgittern\\
                  $\rightarrow$ Pseudospin; $\vec\sigma$-zweidimensionale Pauli-Matrix, $\vec k $-Wellenvektor
            \item Spin der $e^-$ erhält extra Term im Hamiltonian
            \item Quantenelektrodynamische Phenomäne meist proportional zu $c$
            \item \textbf{Quantenelektrodynamische Phenomäne dominieren Spin-Effekte in Graphen!}
          \end{itemize}
    }
  \end{frame}

\end{document}
